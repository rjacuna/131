\documentclass{article}
\usepackage{fontspec}
\usepackage{xcolor}
\usepackage{amsmath}
\usepackage{amssymb}
\usepackage{unicode-math}
\usepackage[normalem]{ulem} %Underline
\setmainfont{Times New Roman}
\setmathfont{Latin Modern Math}
\setlength\parindent{0pt}

\usepackage{siunitx}
\usepackage{graphicx}
\usepackage[justification=centering]{caption}
\usepackage{pgfplots}
\usepackage{pgfplotstable}
\usepackage{booktabs}
\usepackage{array}
\usepackage{colortbl}
\usepackage{tikz}
\usepackage{makecell}

\begin{document}

\begin{center}
  \textbf{MATH} 131---\textbf{HOMEWORK} 3\\
  \color{red}R\color{teal}icardo
  \color{red}J\color{cyan}.
  \color{red}A\color{teal}cu$\color{red}{\widetilde{\color{teal}\text{n}}}$\color{teal}a\color{black}\\
  \color{teal}(\color{red}862079740\color{teal})\color{black}\\
\end{center}

Q1\quad Find the plane equation of the subspace
$Span((1, 1, 0), (0, 0, 1)$) in $\mathbb{R}^3$:\\

Since $(1,1,0)\cdot(0,0,1)=0$, we can see $(1,1,0) \neq k(0,0,1)
\forall k \in \mathbb{R}$. So, $B = {(1,1,0),(0,0,1)}$ is a basis for
the subspace. The plane containing the subspace can be characterized
by its normal vector. Since $(1,1,0)$ and $(0,0,1)$, are two non-zero vectors
in the plane,\\
$ (1,1,0) \times (0,0,1)$
$ =$
$\begin{vmatrix}
  \textbf{i} & \textbf{j} & \textbf{k}\\
  1&1&0\\
  0&0&1
\end{vmatrix}
$$=(1,-1,0)$.\\
So, we get the family of planes $\pi_d := x -y +z +d = 0$
generated by the normal vector $(1,-1,0)$, we know that $Span(B)$ is a
subspace of $\mathbb{R}^3$, so $(0,0,0)$ must be a point in the
plane. So, we can solve for d, by setting $x = y = z = 0 \Rightarrow d
= 0$. Thus, the plane equation becomes $ x -y +z = 0$.\\

Q2\quad Show that if $v_1 , . . . , v_m$ and $w_1 , . . . , w_n$ are vectors in V , then
$Span(v_1 , . . . , v_m ) + Span(w_1 , . . . , w_n ) = Span(v_1 ,
. . . , v_m , w_1 , . . . , w_n)$.\\

\uwave{pf\enskip .}\\
$\forall v \in Span(v_1 , . . . , v_m ): v  = \sum_{i=1}^{m} a_iv_i,
\{a_i\} \subset \mathbb{F}, i \in [1,m] \subset \mathbb{N} \\
$
$and\\
\forall w \in Span(w_1 , . . . , w_n ): w = \sum_{i=1}^{n} b_iw_i,
\{b_i\} \subset \mathbb{F}, i \in [1,n] \subset \mathbb{N} \\$
$\Rightarrow v + w = \sum_{i=1}^{m} a_iv_i +\sum_{i=1}^{n} b_iw_i$\\
So, any arbitrary $v+w$ is in the $Span(v_1 ,
. . . , v_m , w_1 , . . . , w_n)$, because it is a linear combination
of the $vectors \{v_1 ,. . . , v_m , w_1 , . . . , w_n\}$.\\
$\Rightarrow lhs \subseteq rhs$\\

$\forall s \in Span(v_1 ,
. . . , v_m , w_1 , . . . , w_n):
s = \sum_{i=1}^{m} a_iv_i +\sum_{i=1}^{n} b_iw_i$\\
Chose, $\{a_i\} = \{0\} \Rightarrow s_1 = 0 +\sum_{i=1}^{n} b_iw_i =
\sum_{i=1}^{n} b_iw_i \in Span(w_1, ... ,w_{n})$, so we can always
chose vectors type $s_1$ in the $Span(w_1, ... ,w_n)$, for some
$\{b_i\}_{i=1}^n \subset \mathbb{F}$. Similarly, choosing
$\{b_i\} = \{0\}$ gives vectors type $s_2 \in Span(v_1, ... ,v_m)$,
for some $\{a_i\}_{i=1}^m \subset \mathbb{F}$.\\
So,\\
we can always find vectors type $s_3 = s_1 + s_2 \in Span(v_1, ... , v_n)+ Span(w_1, ... ,w_n)$.\\
$\Rightarrow rhs \subseteq lhs$\\
$\Rightarrow lhs = rhs$\\
\begin{flushright}$\blacksquare$\end{flushright}
\newpage
Q3\quad Explain why no set of four polynomials spans
$(P_4\mathbb{F})$.\\

$P_4(\mathbb{F}) = \{p(t) = a_0 + a_1 t + a_2 t^2 + a_3 t^3+ a_4 t^4|\enskip
i \in [0,4] \subset \mathbb{N}; \{a_i\} \in \mathbb{F}\}$\\

\uwave{pf\enskip .}\\
So, a natural basis for $P_4(\mathbb{F})$ is $B = \{
1,t,t^2,t^3,t^4\}$,\\ that is $Span(B) = \sum_{i=0}^4 a_it^i =
P_4(\mathbb{F})$. This fact can be found on `2A.pdf', Example 5. Notice, that the number of elements in $B$ is
equal to 5. So, any set of polynomials $p(t)$, with less that 5
elements cannot Span $P_4(\mathbb{F})$. Because, by Proposition 2 on
the same file which reads `In a finite-dimensional vector space, the length of
every linearly independent set
is smaller than or equal to the length of every spanning set'. And $4 < 5$ so, no set of
four polynomials can span $P_4(\mathbb{F})$, because if the set is
linearly independent, then it cannot span it. If the set is not
linearly dependent, one of the polynomials can be expressed as a
linear combination of the others, so it can be removed without
changing the span of the set, so the new set $B^\prime$ of 3 elements
also cannot $span P_4(\mathbb{F})$, and doesn't even have four elements.
\begin{flushright}$\blacksquare$\end{flushright}\\

Q4\quad Prove or give a counterexample: Let $W$ and $U$ are two subspaces
of $V$ and $x \in V$. If $x \notin W$ and $x \notin U$, then $x \notin
W+U$.\\

Consider the contrapositive statement:\\
If not($x \notin W+U$), then not ($x \notin W$ and $x \notin U$)\\
Now, Double Negation:\\
If $x \in W+U$, then not ($x \notin W$ and $x \notin U$)\\
by DeMorgan's Law, the statement becomes:\\
If $x \in W+U$, then not ($x \notin W$) or not ($x \notin U$)\\
Which becomes by two applications of Double Negation:\\
If $x \in W+U$, then $x \in W$ or $x \in U$\\

Suppose $V = \mathbb{R}^3$, $W \{= (k,0,0) \in \mathbb{R^3}| k \in
\mathbb{R}\}$ and
$U \{= (0,s,0) \in \mathbb{R}^3| s \in \mathbb{R}\}$.\\
So, $W+U = \{(k,s,0) \in \mathbb{R}^3| k,s \in \mathbb{R}\}$.\\

So, clearly this is false, take $(1,1,0) \in W+U$, that is not in $W$,
and it is also not in $U$. So, the original statement can't be true,
since $(1,1,0) \notin W$ and $(1,1,0) \notin U$, but is actually in $W+U$.

\newpage

Q5\quad Suppose $\{v_1 , v_2 , v_3 , v_4 \}$ is linearly independent in V. Prove or give
a counterexample: the subset
$\{v_1 − v_2 , v_2 − v_3 , v_3 − v_4 , v_4 \}$
is also linearly independent.\\

\uwave{pf\enskip .}\\
$\{v_1 , v_2 , v_3 , v_4\}$ is linearly independent in V
$\Rightarrow c_1v_1+ c_2v_2 + c_3v_3  +c_4v_4 = 0$ (1) \\, has only the
trivial solution $c_1=c_2=c_3=c_4=0$, for some scalars
$c_1,c_2,c_3,c_4 \in \mathbb{F}$.\\

Now, set up the dependence test equation, for some scalars $b_1,b_2,b_3,b_4 \in \mathbb{F}$:\\
$b_1(v_1 − v_2) +b_2(v_2 − v_3) +b_3(v_3 − v_4) +b_4v_4 = 0$\\
$\Rightarrow b_1v_1 −b_1v_2 +b_2v_2 −b_2v_3 +b_3v_3 −b_3v_4 +b_4v_4 =
0$\\, By the distributive law when vectors are added in Definition 1 property 7 in file `1B.pdf'.\\
$\Rightarrow b_1v_1 +(b_2 −b_1)v_2 +(b_3 −b_2)v_3 +(b_4 −b_3)v_4 =
0$ (2)\\, Since addition is commutative in the field $\mathbb{F}$, and by the distributive law when scalars are added in Definition 1
property 8 in file `1B.pdf'.

Notice, equation (1) implies, equation (2) has the same values for the
scalars:\\ $b_1 = c_1 = 0$, $b_2-b_1 = c_2 = 0$, $b_3 -b_2 = c_3 = 0$,
$b_4 - b_3 = c_4 =0$.\\
$\Rightarrow b_2 -b_1 =b_2 -0 =b_2 =0$ \\and $b_3 - b_2 = b_3 - 0 = b_3
= 0$ \\and $b_4 -b_3 = b_4 - 0 = b_4 = 0$\\
$\Rightarrow \{v_1 − v_2 , v_2 − v_3 , v_3 − v_4 , v_4 \}$ is also
linearly independent.
\begin{flushright}$\blacksquare$\end{flushright}\\

Q6\quad Prove or give a counterexample:
If $\{v_1 , . . . , v_m\}$ and $\{w_1 , . . . , w_m\}$ are linearly independent subsets of vectors in $V$ , then
${v 1 + w 1 , . . . , v m + w m }$ is linearly independent.\\

Take $V = \mathbb{R}^2$, $\{(1,0)\}$ and $\{(-1,0)\}$, are both
linearly independent, because they are singleton sets, and their
element is not the $0$ vector. $\{(1,0)+(-1,0)\} = \{(0,0)\}$. Is not
linearly independent, since $a(0,0)=(0,0)$, has infinitely many non zero
solutions $a \in \mathbb{R}$, in particular $a=1$.
\end{document}

%%% Local Variables:
%%% mode: latex
%%% TeX-master: t
%%% End:
