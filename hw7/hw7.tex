\documentclass{article}
\usepackage{fontspec}
\usepackage{xcolor}

\usepackage{amsthm}
\usepackage{amsmath}
\usepackage{amssymb}
\usepackage{unicode-math}
\usepackage[makeroom]{cancel}

\usepackage[normalem]{ulem}

\setmainfont{Times New Roman}
\setmathfont{Latin Modern Math}

\setlength\parindent{0em}
\setlength\parskip{0.618em}
\usepackage[a4paper,lmargin=1in,rmargin=1in,tmargin=1in,bmargin=1in]{geometry}

\usepackage{enumitem}

\renewcommand\qedsymbol{$\blacksquare$}

\begin{document}

\begin{center}
  \textbf{MATH} 131---\textbf{HOMEWORK} 7

  \color{red}R\color{teal}icardo
  \color{red}J\color{cyan}.
  \color{red}A\color{teal}cu$\color{red}{\widetilde{\color{teal}\text{n}}}$\color{teal}a\color{black}

  \color{teal}(\color{red}862079740\color{teal})\color{black}
\end{center}\vspace{1.618em}

\paragraph{Q1} Suppose $U$,$V$ and $W$ are finite-dimensional vector
spaces and $S \in \mathcal{L}(V,W)$ and $T \in
\mathcal{L}(U,V)$. Prove that
\[\text{dim(range}\enskip ST) \leq min\{\text{dim(range}\enskip
  S), \text{dim(range}\enskip T)\}\]
\uwave{Pf\enskip.}
\vspace{0.618 em}

Let $U$,$V$ and $W$ be finite-dimensional vector spaces
spaces and $S \in \mathcal{L}(V,W)$ and $T \in
\mathcal{L}(U,V)$.

$ST: U \rightarrow W$ defined by $ST(u) = S(T(u))$, $u \in U$

range $ST$ = $ST(U) = S(T(U)) = S$$($range $T)$

$\Rightarrow$ range $ST = \{\mathrm{w} \in W | \exists \mathrm{v} \in$ range $T:
\mathrm{w} = S(v)\}$

$\Rightarrow$ range $ST \trianglelefteq$ range $S$

$\Rightarrow$ dim$($range $ST) \leq$ dim$($range $S)$ (1)

Let $B_{T(U)} = \{v_i\}_{i=1}^n$ be a basis for range $T$,

then
$B_{S(T(U))} = \{S(v_i)\}_{i=1}^n$ spans range $ST$ (by Q2 HW6)

If $T$ is surjective, then range $T = V$,

so range $ST = S(V) =$ range $S$

If $T$ is not surjective,
then either $B_{S(T(U))}$ is linearly
independent or not.

If $B_{S(T(U))}$ is linearly independent, then dim$($range $ST)
= n =$ dim$($ range $T)$

If $B_{S(T(U))}$ is not linearly independent, then we can reduce it
to a basis of range $ST$,

accordingly dim$($range $ST) < n =$ dim$($ range $T)$

In both cases, dim$($range $ST) \leq $ dim$($ range $T)$ (2)

If dim$($range $S) <$ dim$($range $T)$, then  dim$($range $ST) \leq$
dim$($range $S)$, which is always true by (1).

If dim$($range $T) \leq$ dim$($range $S)$, then  dim$($range $ST) \leq$
dim$($range $T)$, by (2).

So, dim$($range $ST) \leq min\{$ dim$($ range $T)$, dim$($range
$S)\}$.
\vspace{0.618 em}

$\blacksquare$


\newpage
\paragraph{Q2} Suppose that $V$ is finite-dimensional and $T\in \mathcal{L}(V, W )$. Prove
that $T$ is injective if and only if there exists $S\in \mathcal{L}(W,V)$ such that $ST$ is the identity map
on $V$.

\uwave{Pf\enskip.}
\vspace{0.618 em}

$\forall V , W :$ dim$(V) \in \mathbb{N}: T\in \mathcal{L}(V, W)$

$(\Rightarrow)$ Assume $T$ is injective

$\Rightarrow$ null $T$ $= \{0\}$

$\Rightarrow$ dim$(V) =$ dim$($null $T) +$ dim$($range $T)$

$=$ dim$(\{0\}) +$ dim$($range $T)$

$=$ $ 0$ $+$ dim$($range $T) = $ dim$($range $T)$

$\Rightarrow$ dim$(V) = m =$ dim$($range $T)$ (0)

Let $B_0 = \{v_i\}_{i=1}^m$, be a basis for $V$,

then $B = \{T(v_i)=w_i\}_{i=1}^m$, is a basis for range $T$ (by 0)

One can extend $B$ to a basis $B^\prime = \{w_i\}_{i=1}^n$, $m \leq n$

Define, $S: W \rightarrow V$ by $\begin{cases}
  S(w_i) = v_i, 1 \leq i \leq m\\
  S(w_i) = 0, m < i \leq n
\end{cases}$

$ST(v_i) = S(T(v_i)) = S(w_i) = v_i = id_V(v_i), 1 < i < m$

$\Rightarrow \exists S \in \mathcal{L}(W,V): ST = id_V$

$(\Leftarrow)$ Assume $\exists S \in \mathcal{L}(W,V): ST = id_V$

$\forall v_x,v_y \in V :$ $T(v_x) = T(v_y)$

$\Rightarrow S(T(v_x)) = S(T(v_y))$

$\Rightarrow ST(v_x) = ST(v_y)$

$\Rightarrow id_V(v_x) = id_V(v_y)$

$\Rightarrow v_x = v_y$

$\Rightarrow$ $T$ is injective

So,

$T$ is injective $\Leftrightarrow \exists S \in \mathcal{L}(W,V):
ST = id_V$

\vspace{0.618 em}
$\blacksquare$


\newpage

\paragraph{Q3} Suppose $T \in L(P_2(\mathbb{R}), P_4(\mathbb{R}))$ is
the linear map defined by

\[T p = x^2 p.\]

(1) Find the matrix of $T$ with respect to the standard basis.

$\mathcal{E} = \{1, x, x^2\}$

$\mathcal{F} = \{1, x, x^2, x^3, x^4\}$

$T 1 = x^2 1 = x^2 \begin{bmatrix} 0\\ 0 \\ 1 \\ 0 \\ 0\\\end{bmatrix}_{\mathcal{F}}$
$T x = x^2 x = x^3 =\begin{bmatrix} 0\\ 0 \\ 0 \\ 1 \\
  0\\\end{bmatrix}_{\mathcal{F}}$
$T x^2 = x^2 x^2 = x^4 = \begin{bmatrix} 0\\ 0 \\ 0 \\ 0 \\
  1\\\end{bmatrix}_{\mathcal{F}}$

$\Rightarrow [T]_{\mathcal{F} \leftarrow \mathcal{E}} =
\begin{bmatrix} 0& 0 &0 \\ 0&0&0 \\ 1&0&0 \\ 0&1&0 \\ 0&0&1\\\end{bmatrix}$

(2) Verify the fundamental theorem of linear maps.

range $T =$ span $\{x^2,x^3,x^4\}$

null $T =$ span $\{0\}$, since $x^2p = 0$, has only solution $p = 0,
\forall x$

dim $P_2(\mathbb{R}) = |\mathcal{E}| = 3 = 0 + 3 =$ dim $\{0\} +$ dim
$\{x^2,x^3,x^4\}$ $\checkmark$

\newpage
\paragraph{Q4} Let $S, T ∈ \mathcal{L}(V, W)$ and $λ ∈ F$. Let $\mathcal{E} = \{e_1 , ... , e_n\}$ be a basis
of $V$, and $\mathcal{F} = \{f_1 , ... , f_m \}$ be a basis of $W$. Show that there are identities of matrices as
following:
\[[S + T ]_{\mathcal{F} \leftarrow \mathcal{E}} = [S]_{\mathcal{F}
    \leftarrow \mathcal{E}} + [T ]_{\mathcal{F} \leftarrow \mathcal{E}},\]
and
\[[λS]_{\mathcal{F} \leftarrow \mathcal{E}} = λ[S]_{\mathcal{F}
    \leftarrow \mathcal{E}}.\]

\uwave{Pf\enskip.}
\vspace{0.618 em}

Let $S, T ∈ \mathcal{L}(V, W)$ and $λ ∈ \mathbb{F}$. Let $\mathcal{E} = \{e_1 , ... , e_n\}$ be a basis
of $V$, and $\mathcal{F} = \{f_1 , ... , f_m \}$be a basis of $W$.

$\exists a_{ji} \in \mathbb{F}: S(e_i) = \sum_{j=1}^ma_{ji}f_j$
$\Rightarrow [S]_{\mathcal{F} \leftarrow \mathcal{E}} =
$
$\begin{bmatrix}
a_{11}&\cdots & a_{1n}\\
\vdots&\ddots&\vdots\\
a_{m1}&\cdots& a_{mn}
\end{bmatrix}
$

and

$\exists b_{ji} \in \mathbb{F}: T(e_i) = \sum_{j=1}^mb_{ji}f_j$
$\Rightarrow [T]_{\mathcal{F} \leftarrow \mathcal{E}} =
\begin{bmatrix}
b_{11}&\cdots & b_{1n}\\
\vdots&\ddots&\vdots\\
b_{m1}&\cdots& b_{mn}
\end{bmatrix}
$

$\Rightarrow [S]_{\mathcal{F} \leftarrow \mathcal{E}} + [T]_{\mathcal{F} \leftarrow \mathcal{E}} =$
$\begin{bmatrix}
a_{11}&\cdots & a_{1n}\\
\vdots&\ddots&\vdots\\
a_{m1}&\cdots& a_{mn}
\end{bmatrix}
$$ + $
$\begin{bmatrix}
b_{11}&\cdots & b_{1n}\\
\vdots&\ddots&\vdots\\
b_{m1}&\cdots& b_{mn}
\end{bmatrix}
$$ = $
$\begin{bmatrix}
a_{11} + b_{11} &\cdots & a_{1n} + b_{1n}\\
\vdots&\ddots&\vdots\\
a_{m1} + b_{m1}&\cdots& a_{mn} + b_{mn}
\end{bmatrix}
$




$(S+T)(e_i) := S(e_i) + T(e_i) = \sum_{j=1}^ma_{j,i}f_j +
\sum_{j=1}^mb_{j,i}f_j = \sum_{j=1}^m(a_{j,i} + b_{j,i})f_j$


$\Rightarrow [S+T]_{\mathcal{F} \leftarrow \mathcal{E}} =$
$\begin{bmatrix}
a_{11} + b_{11} &\cdots & a_{1n} + b_{1n}\\
\vdots&\ddots&\vdots\\
a_{m1} + b_{m1}&\cdots& a_{mn} + b_{mn}
\end{bmatrix}
$

$\Rightarrow [S + T ]_{\mathcal{F} \leftarrow \mathcal{E}} = [S]_{\mathcal{F}
    \leftarrow \mathcal{E}} + [T ]_{\mathcal{F} \leftarrow
    \mathcal{E}}$\\

$\lambda S(e_i) = \lambda \sum_{j=1}^ma_{j,i}f_j = \sum_{j=1}^m
\lambda a_{j,i}f_j$

Also,

$\Rightarrow [\lamda S]_{\mathcal{F} \leftarrow \mathcal{E}} =$
$\begin{bmatrix}
\lambda a_{11}&\cdots & \lamda a_{1n}\\
\vdots&\ddots&\vdots\\
\lambda a_{m1}&\cdots& \lambda a_{mn}
\end{bmatrix}
$

and

$[S]_{\mathcal{F} \leftarrow \mathcal{E}} =$
$\begin{bmatrix}
a_{11}&\cdots & a_{1n}\\
\vdots&\ddots&\vdots\\
a_{m1}&\cdots& a_{mn}
\end{bmatrix}
$
$\Rightarrow \lambda [S]_{\mathcal{F} \leftarrow \mathcal{E}} =$
$\lambda
\begin{bmatrix}
a_{11}&\cdots & a_{1n}\\
\vdots&\ddots&\vdots\\
a_{m1}&\cdots& a_{mn}
\end{bmatrix}
$$ = $
$\begin{bmatrix}
\lambda a_{11}&\cdots & \lamda a_{1n}\\
\vdots&\ddots&\vdots\\
\lambda a_{m1}&\cdots& \lambda a_{mn}
\end{bmatrix}
$

$\Rightarrow [λS]_{\mathcal{F} \leftarrow \mathcal{E}} = λ[S]_{\mathcal{F}
    \leftarrow \mathcal{E}}$


\vspace{0.618 em}
$\blacksquare$

\newpage
\paragraph{Q5} Suppose $D ∈ \mathcal{L}(P_3(\mathbb{R}),
P_2(\mathbb{R}))$ is the differential map defined by \[Dp = p^\prime\]

Find a basis $\mathcal{E}$ of $P_3(\mathcal{R})$ and a basis
$\mathcal{F}$ of $P_2(\mathcal{R})$ such that the matrix of D with
respect  to these bases is

\[ [D]_{\mathcal{F} \leftarrow \mathcal{E}} =
  \begin{pmatrix}
    1 & 0& 0& 0\\
    0 & 1& 0& 0\\
    0 & 0& 1& 0
  \end{pmatrix}
\]


$Dx = 1$, $D\frac{1}{2}x^2 = x$, $D\frac{1}{3}x^3 = x^2$, $D1 = 0$

So, $\mathcal{E} = \{x,\frac{1}{2}x^2,\frac{1}{3}x^3,1\}$, and
$\mathcal{F} = \{1,x,x^2\}$,
or $\mathcal{E} = \{x,x^2,x^3,1\}$, and
$\mathcal{F} = \{1,2x,3x^2\}$,

or
$\mathcal{E} = \{2000 + x, 2000 + x^2, 2000 + x^3,5\}$, and
$\mathcal{F} = \{1,2x,3x^2\}$

Really, bunch, infinitely many.


\paragraph{Q6} Find linear maps $S, T ∈ \mathcal{L}(\mathbb{R}^2)$
such that $ST \neq TS$.

$S , T \in \mathcal{L}(\mathbb{R}^2): \{e_1,e_2\}$ is the standard
basis for $\mathbb{R}^2$

$S: \mathbb{R}^2 \rightarrow \mathbb{R}^2$ defined by

$S(e_1) = e_1$

$S(e_2) = 2e_2$

$T: \mathbb{R}^2 \rightarrow \mathbb{R}^2$ defined by

$T(e_1) = e_2$

$T(e_2) = 2e_1$

$\Rightarrow ST(e_1) = S(T(e_1)) = S(e_2) = 2e_2$ and $ST(e_2) =
S(T(e_2)) = S(2e_1) = 2S(e_1)= 2e_1$

$\Rightarrow TS(e_1) = T(S(e_1)) = T(e_1) = e_2$ and $TS(e_2) =
T(S(e_2)) = T(2e_2) = 2T(e_2)= 2(2e_1) = 4e_1$

$\Rightarrow ST \neq TS$

\end{document}

%%% Local Variables:
%%% mode: latex
%%% TeX-master: t
%%% End:
