\documentclass{article}
\usepackage{fontspec}
\usepackage{xcolor}

\usepackage{amsthm}
\usepackage{amsmath}
\usepackage{amssymb}
\usepackage{unicode-math}
\usepackage[makeroom]{cancel}

\usepackage[normalem]{ulem}

\setmainfont{Times New Roman}
\setmathfont{Latin Modern Math}

\setlength\parindent{0em}
\setlength\parskip{0.618em}
\usepackage[a4paper,lmargin=1in,rmargin=1in,tmargin=1in,bmargin=1in]{geometry}

\usepackage{enumitem}

\renewcommand\qedsymbol{$\blacksquare$}

\DeclareMathOperator*{\bigplus}{\scalerel*{+}{\textstyle\sum}}
\usepackage{scalerel}

\begin{document}

\begin{center}
  \textbf{MATH} 131---\textbf{HOMEWORK} n

  \color{red}R\color{teal}icardo
  \color{red}J\color{cyan}.
  \color{red}A\color{teal}cu$\color{red}{\widetilde{\color{teal}\text{n}}}$\color{teal}a\color{black}

  \color{teal}(\color{red}862079740\color{teal})\color{black}
\end{center}\vspace{1.618em}

``$\guillemotright n \guillemotleft$'' $:= $ ``Statement number $n$''

\paragraph{Q1} Suppose $V$ is finite-dimensional, with dim $V = n ≥ 1$. Prove that
there exist 1–dimensional subspaces $U_1 , ... , U_n$ of $V$ such that
\[V = U_1 ⊕ ··· ⊕ U_n.\]

\uwave{Pf\enskip.}
\vspace{0.618 em}

$\forall V:$ dim $V = n \geq 1, n \in \mathbb{N}$

Let $\mathcal{U}:=\{u_i\}_{i=1}^n$ be a basis of $V$

$\forall u_i \in \mathcal{U}: U_i :=$ span$(\{u_i\})$

$\{u_i\}$ is a basis for $U_i$ by construction $\Rightarrow$ dim $U_i
= |\{u_i\}| = 1$

if $i = j$, then $U_i \cap U_j = U_i$

if $i \neq j$, then $U_i \cap U_j =$ span $(\{u_i\}) \cap$ span
$(\{u_j\})$

$= \{v \in V| v \in
$span $(\{u_i\})$ and $ v \in$ span $(\{u_j\})\}$

$\Rightarrow \exists a,b\in \mathbb{F}: v = a u_i = b u_j$ $\Rightarrow a u_i - b u_j = 0$

Since, $u_i, u_j \in \mathcal{U}$ and $\mathcal{U}$  is a basis for
$V$.

It follows that,
$a = b = 0$ in the dependence test equation above.

$\Rightarrow v = 0 \Rightarrow U_i \cap U_j = \{0\}$

Since the spaces intersect pairwise on $\{0\}$ as shown,

and $\forall
i: 0 \in
U_i$, because $U_i =$ span $(\{u_i\})$.

It follows $\bigcap\limits_{i=1}^n U_i = \{0\}$
$\Rightarrow \bigplus\limits_{i=1}^n U_i = \bigoplus\limits_{i=1}^n U_i$

Now, by definition

$\bigplus\limits_{i=1}^n U_i = \{\sum_{i=1}^n
s_i|\enskip s_i \in U_i\}$

$=\{\sum_{i=1}^n s_i |\enskip s_i \in$ span $(\{u_i\})\}$
$=\{\sum_{i=1}^n a_iu_i|\enskip a_i \in \mathbb{F}\}$
$=$ span $(\mathcal{U}) = V$

So, to sum up. for any basis, the span of each singleton subset of the
basis is one dimensional. The intersection of the generated spans is
$\{0\}$, because pairwise it is so, so their sum as subspaces is a direct
sum. And, their direct sum spans the ambient space. So, there exist
1-dimensional subspaces of $V$ such that
\[V = U_1 ⊕ ··· ⊕ U_n.\]

\vspace{0.618 em}
$\blacksquare$

\newpage
\paragraph{Q2} Suppose that $U$ and $V$ are subspaces of $\mathbb{R}^8$ such that dim $U = 3$,
dim $W = 5$, and $U + W = \mathbb{R}^8$. Prove that $\mathbb{R}^8 = U ⊕ W$.

\uwave{Pf\enskip.}
\vspace{0.618 em}

$\forall$ $U,$ $V \trianglelefteq$ $\mathbb{R}^8:$ dim $U = 3$,
dim $W = 5$, and $U + W = \mathbb{R}^8$

Since, dim $(\mathbb{R}^8) = 8$ and $8 \in \mathbb{N}$

dim $U + W =$ dim $U +$  dim $W +$  dim $U \cap W$ (by Theorem 2
in 2C)

$\Rightarrow 8 = 3 + 5 +$ dim $U \cap W$

$\Rightarrow 8 = 8 +$ dim $U \cap W$

$\Rightarrow 8 -8 = $ dim $U \cap W$

$\Rightarrow 0 = $ dim $U \cap W$

$\Rightarrow$ $U \cap W = \{0\}$

$\Rightarrow$ $\mathbb{R}^8 = U \oplus W$

\vspace{0.618 em}
$\blacksquare$



\paragraph{Q3} Prove or give a counterexample:
if $ U_1 , U_2 , W$ are subspaces of $V$ such that
\[U_1 ⊕ W = U_2 ⊕ W,\]
then $U_1 = U_2$.\\


span $(\{(1,0)\}) +$ span $(\{(0,1)\}) = \mathbb{R}^2$

and

span $(\{(1,0)\}) \cap$ span $(\{(0,1)\}) = \{(0,0)\}$ Clearly, just
look at the intersection of the x and the y axes

$\Rightarrow$ span $(\{(1,0)\}) \oplus$ span $(\{(0,1)\}) =
\mathbb{R}^2$ $\guillemotright -1 \guillemotleft$\\

span $(\{(1,-1)\}) +$ span $(\{(0,1)\}) = \mathbb{R}^2$

span $(\{(1,-1)\}) \cap$ span $(\{(0,1)\}) = \{ v \in \mathbb{R}^2
|\enskip v \in$ span $(\{(-1,1)\})$ and $v \in$ span $(\{(0,1)\})\}$

$\Rightarrow$ $\exist a,b\in \mathbb{R}: v = b(-1,1) = a(0,1)$

$\Rightarrow$ $-b=0$ and $b = a$ $\Rightarrow v = (0,0)$

$\Rightarrow$ span $(\{(-1,1)\}) \cap$ span $(\{(0,1)\}) = \{(0,0)\}$

$\Rightarrow$ span $(\{(1,-1)\}) \oplus$ span $(\{(0,1)\}) = \mathbb{R}^2$ $\guillemotright 1 \guillemotleft$

$\guillemotright -1 \guillemotleft$ and $\guillemotright 1
\guillemotleft$ $\Rightarrow$ span $(\{(1,-1)\}) \oplus $span
$(\{(0,1)\}) =$ span $(\{(1,0)\}) \oplus$ span $(\{(0,1)\})$

$(-1,1) \notin$ span $(\{(1,0)\})$ $\Rightarrow$ span $(\{(1,0)\})
\neq$ span $(\{(-1,1)\})$

So, by counterexample Q3 is false

\newpage
\paragraph{Q4} Suppose $U = \{(x, x, y, y) ∈ \mathbb{R}^4 | x, y ∈ \mathbb{R}\}$. Find a subspace $W$ of $\mathbb{R}^4$ such that
\[\mathbb{R}^4 = U ⊕ W\].


For this problem I use facts from Linear Algebra Done Right (LADR), by
Sheldon Axler.

Define $<\mathbf{x},\mathbf{y}> := \sum_{i=1}^4x_iy_i$.
We say that if $<\mathbf{x},\mathbf{y}> = 0$, $\mathbf{x}$ is orthogonal to
$\mathbf{y}$ (by 6.11 in LADR).

$U^\perp = \{v\in V| < v, u> = 0$ for every $u \in U\}$ is called the
orthogonal complement of $U$ (by 6.45 in
LADR)

It's easy to see that, $U =$ span$(\{(1,1,0,0),(0,0,1,1)\})$

Form the coefficient matrix of $U$, with the elements of $U$ as row
vectors as such:\\
$\begin{pmatrix}
  1&1&0&0\\
  0&0&1&1\\
\end{pmatrix}$

This coefficient matrix is already in RREF, so I can read it's
nullspace.

From
column two, I read $(-1,1,0,0)$, and from column 4, I read $(0,0,-1,1)$.

I claim $U^\perp =$ span$(\{(-1,1,0,0),(0,0,-1,1)\})$,\\
since $\{(1,1,0,0),(0,0,1,1)\}$ and
$\{(-1,1,0,0),(0,0,-1,1)\}$\\ are bases for $U$ and $U^\perp$
respectively. It's enough to check that:

$<(1,1,0,0),(-1,1,0,0)> = 0$ and\\
$<(0,0,1,1),(-1,1,0,0)> = 0$ and\\
$<(1,1,0,0),(0,0,-1,1)> =
0$ and\\
$<(0,0,1,1),(0,0,-1,1)> = 0$.

So, $\mathbb{R}^4 = U \oplus U^\perp$  (by 6.47 in LADR)



\paragraph{Q5} Let $U = \{p ∈ P_4(\mathbb{R}) : p^{\prime\prime}(4) = 0\}$. (In homework 5, we have
computed a basis of $U$ and extended it to a basis of $P_4(\mathbb{R})$).
Find a subspace $W$ of $P_4(\mathbb{R})$ such that $P_4(\mathbb{R}) = W ⊕ U$. Justify your answer.

I dont' like my answer, so I'm using the bases Susan used in class:

$\mathcal{U} = \{1,t-4,(t-4)^3,(t-4)^4\}$ a basis for $U$\\
$\mathcal{P} = \{1,t-4,(t-4)^2,(t-4)^3,(t-4)^4\}$ a basis for
$P_4(\mathbb{R})$

Let $\mathcal{W} = \mathcal{P} \backslash \mathcal{U} = \{(t-4)^2\}$:
$W$ = span $\mathcal{W}$

$W \cap U = \{p \in P_4(\mathbb{R})| p \in W$ and $p \in U\}$\\
$\Rightarrow \exists a_i \in \mathbb{R}: p = a_2(t-4)^2= a_0(1)+a_1(t-4)+a_3(t-4)^3+a_4(t-4)^4$\\
$\Rightarrow 0 = -a_2(t-4)^2+ a_0(1)+a_1(t-4)+a_3(t-4)^3+a_4(t-4)^4$

Since, $\mathcal{P}$ is a basis for $P_4(\mathbb{R})$, all $a_i$ must
be $0$ in the dependence test equation above.

So, $p=0 \Rightarrow W \cap U = \{0\}$ $\guillemotright 23 \guillemotleft$

Since, adding linear combinations of vectors
in $\mathcal{W}$ to linear combinations of vectors in $\mathcal{U}$,\\
you get linear combinations of vectors in $\mathcal{P}$.\\
$W + U =$ span $\mathcal{W} +$  span $\mathcal{U} =$ $P_4(\mathbb{R})$ $\guillemotright 32 \guillemotleft$


$\guillemotright 23 \guillemotleft$ and $\guillemotright 32
\guillemotleft$ $\Rightarrow$ $P_4(\mathbb{R}) =$  $W \oplus U$

\newpage
\paragraph{Q6} Suppose $\phi ∈ \mathcal{L}(V, \mathbb{F})$. Suppose $u
∈ V$ is not in Null $\phi$. Let
$U =$ Span$(u)$. Prove that

\[V = \text{Null } \phi \oplus U\]

\uwave{Pf\enskip.}
\vspace{0.618 em}

Suppose $\phi ∈ \mathcal{L}(V, \mathbb{F})$. Suppose $u
∈ V$ is not in Null $\phi$. Let
$U =$ Span$(u)$.

Null $\phi \cap U = \{v \in V| v \in$ Null$\phi$ and $v \in U\}$

Let $b,c \in \mathbb{F},$ and $u^\prime \in$ Null $\phi: v= bu^\prime= cu$\\
$\Rightarrow$ $\phi(bu^\prime)= \phi(cu)$\\
$\Rightarrow$ $b\phi(u^\prime) = b0 = 0 = c\phi(u)$\\
$u \notin$ Null $\phi$ $\Rightarrow \phi(u) \neq 0 \Rightarrow c = 0
\Rightarrow v = 0$

Null $\phi \cap U = \{0\}$

Range $\phi =$ $\phi(V) = \{x \in \mathbb{F}| \phi(v) = x, v \in V\}$

Either $v = au$ or $v \neq au$, some $a \in \mathbb{F}$\\
If $v=au$, then $\phi(v)=\phi(au) = a\phi(u) \neq 0$, whenever $a \neq 0$, since $u \notin$ Null
$\phi$\\
If $v \neq au$, then $\phi(v) = 0$, since $v \in$ Null $\phi$\\
So, Range $\phi$ $= \phi(V) =$ span$(\{\phi(u)\})$ $\Rightarrow$
$\phi^{-1}(\mathbb{F}) =$ $U$\\
$\Rightarrow$ dim$($Range $\phi) =$ dim$(U)$

dim$(V) =$ dim$($Null $ \phi) +$ dim$($Range $ \phi)$ (by FTLA)\\
$\Rightarrow$ dim$(V) =$ dim$($Null $ \phi) +$ dim$(U)$

dim$($Null $\phi + U) =$ dim$($Null $\phi) +$ dim$(U) +$ dim$($Null $\phi \cap U)$\\
$=$ dim$($Null $\phi) +$ dim$(U) +$ dim$(\{0\})$ \\
$=$ dim$($Null $\phi) +$ dim$(U) + 0$\\
$=$ dim$($Null$\phi) +$dim$(U)$

$\Rightarrow$ dim$(V) =$ dim$($Null $\phi + U)$

$\Rightarrow$ $V =$ Null $\phi + U$

$\Rightarrow$ $V =$ Null $\phi \oplus U$

\vspace{0.618 em}
$\blacksquare$


\end{document}

%%% Local Variables:
%%% mode: latex
%%% TeX-master: t
%%% End:
