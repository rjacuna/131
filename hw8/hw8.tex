\documentclass{article}
\usepackage{fontspec}
\usepackage{xcolor}

\usepackage{amsthm}
\usepackage{amsmath}
\usepackage{amssymb}
\usepackage{unicode-math}
\usepackage[makeroom]{cancel}

\usepackage[normalem]{ulem}

\setmainfont{Times New Roman}
\setmathfont{Latin Modern Math}

\setlength\parindent{0em}
\setlength\parskip{0.618em}
\usepackage[a4paper,lmargin=1in,rmargin=1in,tmargin=1in,bmargin=1in]{geometry}

\usepackage{enumitem}

\usepackage{stackengine}

\renewcommand\qedsymbol{$\blacksquare$}

\begin{document}

\begin{center}
  \textbf{MATH} 131---\textbf{HOMEWORK} 8

  \color{red}R\color{teal}icardo
  \color{red}J\color{cyan}.
  \color{red}A\color{teal}cu$\color{red}{\widetilde{\color{teal}\text{n}}}$\color{teal}a\color{black}

  \color{teal}(\color{red}862079740\color{teal})\color{black}
\end{center}\vspace{1.618em}

``$\guillemotright n \guillemotleft$'' $:= $ ``Statement number $n$''

\paragraph{Q1} Determine the following linear maps of vector spaces over $\mathbb{R}$ are
isomorphism or not. If it is an isomorphism, find its inverse map. (Hint: inverse of
matrices.) If it is not an isomorphism, briefly explain why.

(1) (Rotation by 60$^\circ$)
\[L:\mathbb{R}^2 \rightarrow \mathbb{R}^2\]
\[(x,y) \mapsto (\frac{x}{2} -\frac{\sqrt{3}}{2}y, \frac{\sqrt{3}}{2}x +\frac{1}{2}y)\]

$\mathcal{E}=\{e_1,e_2\}=\mathcal{F}: e_1 = (1,0), e_2 = (0,1):$

$ [ L ]_{\mathcal{F} \leftarrow \mathcal{E}} =$
$\left[
  \begin{array}{c|c}
    L(e_1)&L(e_2)
  \end{array}
\right]
$
$=$
$
\begin{bmatrix}
  \frac{1}{2} & -\frac{\sqrt{3}}{2}\\
   \frac{\sqrt{3}}{2} & \frac{1}{2}
\end{bmatrix}
$

I know that the transpose will work because, $L(e_i) \perp L(e_2)$,
and $|L(e_1)|=|L(e_2)|=1$--- i.e. L is an orthogonal matrix.

$ [ L ]_{\mathcal{E} \leftarrow \mathcal{F}}^T =$
$
\begin{bmatrix}
  \frac{1}{2} & \frac{\sqrt{3}}{2}\\
  -\frac{\sqrt{3}}{2} & \frac{1}{2}
\end{bmatrix}
$
$= [ K ]_{\mathcal{F} \leftarrow \mathcal{E}}$

Just routine verification will show that $[L][K] = I = [K][L]$

$
\begin{bmatrix}
  \frac{1}{2} & -\frac{\sqrt{3}}{2}\\
  \frac{\sqrt{3}}{2} & \frac{1}{2}
\end{bmatrix}
$
$
\begin{bmatrix}
  \frac{1}{2} & \frac{\sqrt{3}}{2}\\
  -\frac{\sqrt{3}}{2} & \frac{1}{2}
\end{bmatrix}
$
$=$
$
\begin{bmatrix}
  (\frac{1}{2})^2 +(-\frac{\sqrt{3}}{2})^2 &
  \frac{1}{2}(\frac{\sqrt{3}}{2}) - \frac{\sqrt{3}}{2}(\frac{1}{2})\\
  \frac{\sqrt{3}}{2}(\frac{1}{2}) + \frac{1}{2}(-\frac{\sqrt{3}}{2}) &
  (\frac{\sqrt{3}}{2})^2 + (\frac{1}{2})^2
\end{bmatrix}
$

$= I =$

$
\begin{bmatrix}
  (\frac{1}{2})^2 +(\frac{\sqrt{3}}{2})^2 &
  \frac{1}{2}(-\frac{\sqrt{3}}{2}) + \frac{\sqrt{3}}{2}(\frac{1}{2})\\
  -\frac{\sqrt{3}}{2}(\frac{1}{2}) + \frac{1}{2}(\frac{\sqrt{3}}{2}) &
  (-\frac{\sqrt{3}}{2})^2 + (\frac{1}{2})^2
\end{bmatrix}
$
$=$
$
\begin{bmatrix}
  \frac{1}{2} & \frac{\sqrt{3}}{2}\\
  -\frac{\sqrt{3}}{2} & \frac{1}{2}
\end{bmatrix}
$
$
\begin{bmatrix}
  \frac{1}{2} & -\frac{\sqrt{3}}{2}\\
  \frac{\sqrt{3}}{2} & \frac{1}{2}
\end{bmatrix}
$

Reading back the linear map corresponding to $[ K ]$ we have, the
inverse map $K$ of $L$:

$K:\mathbb{R}^2 \rightarrow \mathbb{R}^2$ defined by\\
$(x,y) \mapsto (\frac{\sqrt{3}}{2}y+\frac{1}{2}x,\frac{1}{2}y-\frac{\sqrt{3}}{2}x)$

So, $L$ is an isomorphism.

(2) (Reflection about x–axis)
\[L:\mathbb{R}^2 \rightarrow \mathbb{R}^2\]
\[(x, y) \mapsto (x, −y)\]

Immediately $L$ is it's own inverse.

$(x,y) \stackrel{L}{\rightarrow} (x,-y) \stackrel{L}{\rightarrow}
(x,-(-y)) = (x,y)$

So, $L$ is an isomorphism.
\newpage
(3)

\[L:\mathbb{R}^2 \rightarrow \mathbb{R}^2\]
\[(x,y,z) = (x+2y+3z,4x+5y+6z,7x+8y+9z)\]

$\mathcal{E}=\{e_1,e_2,e_3\}=\mathcal{F}: e_1 = (1,0,0), e_2 =
(0,1,0), e_3 = (0,0,1):$

$[ L ]_{\mathcal{F} \leftarrow \mathcal{E}} =$

$\left[
  \begin{array}{c|c|c}
    L(e_1)&L(e_2)&L(e_3)
  \end{array}
\right]
$
$=$
$
\begin{bmatrix}
  1&2&3\\
  4&5&6\\
  7&8&9
\end{bmatrix}
$

Gauss-Jordan Elimination:

$\left[
  \begin{array}{ccc|ccc}
  1&2&3&1&0&0\\
  4&5&6&0&1&0\\
  7&8&9&0&0&1
  \end{array}
\right]
$
$\xrightarrow[-7R_1+R_3 \mapsto R_3]{-4R_1+R_2 \mapsto R_2}
$\left[
  \begin{array}{ccc|ccc}
  1&2&3&1&0&0\\
  0&-3&-6&-4&1&0\\
  0&-6&-12&-7&0&1
  \end{array}
\right]
$
$\xrightarrow[]{-2R_1+R_2 \mapsto R_2}
$\left[
  \begin{array}{ccc|ccc}
  1&2&3&1&0&0\\
  0&-3&-6&-4&1&0\\
  0&0&0&1&0&1
  \end{array}
\right]
$

Since, adding a scalar multiple of a row to another doesn't
change the determinant, this Gauss-Jordan Matrix in Row Echelon Form (REF),
has the same determinant as $[L]$. The determinant of the REF is the
trace of the REF which is $0$, so the determinant of $[L]$ is $0$.

So, $L$ is not invertible.

So, $L$ is not an isomorphism.

\paragraph{Q2} Determine the following spaces are isomorphic or not. If they are
isomorphic, give one isomorphism explicitly.\\

(1) $\mathcal{L}(\mathbb{R}^2 , \mathbb{R}^5)$ and $\mathbb{R}^7.$

(by Example 2 in 3D) $\mathcal{L}(\mathbb{R}^2 , \mathbb{R}^5) \cong$
Mat$_{2\times5}(\mathbb{R})$

dim(Mat$_{2\times5}(\mathbb{R})$) $= 2(5) = 10 \neq 7 =$
dim($\mathbb{R}^7$)

So, $\mathcal{L}(\mathbb{R}^2 , \mathbb{R}^5)$ and $\mathbb{R}^7$ are
not isomorphic (by Theorem 1 in 3D)\\

(2) Span$((1, 1, 0), (2, 5, 6))$ and $\mathbb{R}^3$

Since, $(1,1,0) \neq k(2,5,6)$, where $k \in \mathbb{R}$

$\{(1,1,0),(2,5,6)\}$ is a basis for Span$((1, 1, 0), (2, 5, 6))$.

So, dim(Span$((1, 1, 0), (2, 5, 6))$) $=$ $|\{(1,1,0),(2,5,6)\}| = 2
\neq 3 =$ dim($\mathbb{R}^3$)\\

\newpage
(3) $\{(x, y, z) ∈ \mathbb{R}^3 | 2x + 2y + z = 0\}$ and
$\mathbb{R}^2.$\\

$z=-2x-2y \Rightarrow (x,y,z) \mapsto (x,y,-2x-2y)$

$\Rightarrow (1,0,0) \mapsto (1,0,-2)$,$ (0,1,0) \mapsto (0,1,-2)$, and
$(0,0,1) \mapsto (0,0,0)$

$\Rightarrow$$\{(x, y, z) ∈ \mathbb{R}^3 | 2x + 2y + z = 0\}$ $=$
span($\{(1,0,-2),(0,1,-2)\}$)

$\Rightarrow$dim($\{(x, y, z) ∈ \mathbb{R}^3 | 2x + 2y + z = 0\}$) $=$
$|\{(1,0,-2),(0,1,-2)\}| = 2 =$ dim($\mathbb{R}^2$)

(by Theorem 1 in 3D) they're isomorphic\\


$R: \{(x, y, z) ∈ \mathbb{R}^3 | 2x + 2y + z = 0\} \rightarrow
\mathbb{R}^2$ defined by

$(1,0,-2) \mapsto(1,0)$\\
$(0,1,-2) \mapsto(0,1)$

Extend $R$ to a linear map for any $a_1,a_2 \in \mathbb{R}$ by,\\
$R(a_1(1,0,-2)+a_2(0,1,-2))= a_1R(1,0,-2)+a_2R(0,1,-2)$

$J:\mathbb{R}^2\rightarrow \{(x, y, z) ∈ \mathbb{R}^3 | 2x + 2y + z = 0\}
$ defined by

$(x,y) \mapsto (x,y,-2x-2y)$


$(1,0,-2) \stackrel{R}{\rightarrow} (1,0) \stackrel{J}{\rightarrow}
(1,0,-2) $\\
$(0,1,-2) \stackrel{R}{\rightarrow} (0,1) \stackrel{J}{\rightarrow} (0,1,-2) $

$\Rightarrow JR = id_{ \{(x, y, z) ∈ \mathbb{R}^3 | 2x + 2y + z = 0\}}$


$(1,0) \stackrel{J}{\rightarrow} (1,0,-2) \stackrel{R}{\rightarrow}
(1,0) $\\
$(0,1) \stackrel{J}{\rightarrow} (0,1,-2) \stackrel{R}{\rightarrow}
(0,1) $\\

$\Rightarrow JR = id_{\mathbb{R}^2}$




\paragraph{Q3}Suppose $T ∈ \mathcal{L}(U, V )$ and $S ∈ \mathcal{L}(V, W )$ are both invertible linear
maps. Prove that $ST ∈ \mathcal{L}(U, W )$ is invertible and
$(ST)^{-1} = T^{-1} S^{-1}$.

\uwave{Pf\enskip.}
\vspace{0.618 em}

Suppose $T ∈ \mathcal{L}(U, V)$ and $S ∈ \mathcal{L}(V, W)$ are both invertible linear
maps.\\

$T$ is invertible $\Rightarrow \exists T^{-1} \in \mathcal{L}(V,U):
TT^{-1}=I_{n\times n}=T^{-1}T$, $n \in \mathbb{N}$
$\guillemotright 1 \guillemotleft$

$\Rightarrow$ $T$ is an isomorphism $\Rightarrow$ $U
\cong V \Rightarrow$ dim$(U) =$ dim$(V) := n$ $\guillemotright \alpha \guillemotleft$

$S$ is invertible $\Rightarrow \exists S^{-1} \in \mathcal{L}(W,V):
SS^{-1}=I_{m\times m}=S^{-1}S$, $m \in \mathbb{N}$ $\guillemotright 2 \guillemotleft$

$\Rightarrow$ $S$ is an isomorphism $\Rightarrow$ $V
\cong W \Rightarrow$ dim$(V) = $ dim$(W) := m$ $\guillemotright \beta \guillemotleft$

$\guillemotright \alpha \guillemotleft$ and $\guillemotright \beta \guillemotleft$ $\Rightarrow$ $m = n$ $\Rightarrow$
$I_{n\times n} = I_{m\times m} := I$

$STT^{-1}S^{-1}\stackrel{\guillemotright 1 \guillemotleft}{=\joinrel=} S I
S^{-1}
\stackrel{multiply\enskip by\enskip
  I}{=\joinrel=\joinrel=\joinrel=\joinrel=\joinrel=\joinrel=\joinrel=\joinrel=}
SS^{-1}\stackrel{\guillemotright 2 \guillemotleft}{=\joinrel=}  I\enskip\enskip\enskip \guillemotright 3 \guillemotleft$\\
$T^{-1}S^{-1}ST\stackrel{\guillemotright 2 \guillemotleft}{=\joinrel=} T I
T^{-1}
\stackrel{multiply\enskip by\enskip
  I}{=\joinrel=\joinrel=\joinrel=\joinrel=\joinrel=\joinrel=\joinrel=\joinrel=}
T^{-1}T\stackrel{\guillemotright 1 \guillemotleft}{=\joinrel=}
I\enskip \quad\guillemotright 4 \guillemotleft
$\\

$\guillemotright 3 \guillemotleft$ and $\guillemotright 4 \guillemotleft$ $\Rightarrow (ST)^{-1}=T^{-1}S^{-1}$


\vspace{0.618 em}
$\blacksquare$

\newpage
\paragraph{Q4} Suppose $V$ is a finite-dimensional and $S, T ∈ \mathcal{L}(V)$. Prove that $ST$ is
invertible if and only if both $S$ and $T$ are invertible.

\uwave{Pf\enskip.}
\vspace{0.618 em}

$\forall S,T \in \mathcal{L}(V):$

$(\Leftarrow)$ Let $V=U=W$ in Q3 of this homework and the desired
result is proven. Namely $ST$ is invertible.

$(\Rightarrow)$ Ass. $ST$ is invertible

Consider the contrapositive statement, of ``if $ST$ is invertible, then
both $S$ and $T$ are invertible'',\\ which is ``if not
both $S$ and $T$ are invertible, then $ST$ is not invertible''.\\ By
DeMorgan's Theorem it turns into ``if $S$ is not invertible or $T$ is not
invertible, then $ST$ is not invertible''

So, further assume $S$ is not invertible.

$\Rightarrow $det$([S])=0$

(By, Ass.) $ST$ invertible $\Rightarrow$ det$([ST]) \neq 0 \quad \guillemotright \alpha \guillemotleft$

But, by the properties of the determinant,

det$([ST]) =$ det$([S][T]) =$ det$([S])$det$([T]) = 0$ det$([T]) =0
\quad$ $\guillemotright \beta \guillemotleft$

$\guillemotright \alpha \guillemotleft$ contradicts $\guillemotright \beta \guillemotleft$ so $ST$ cannot be invertible if $S$
is not invertible.\\
By the symmetry of the problem, $ST$ cannot be invertible if $T$ is
not invertible.\\
So, the contrapositive statement is proven. This completes the proof.


\vspace{0.618 em}
$\blacksquare$

\paragraph{Q5} Suppose $V$ is finite-dimensional and dim $V$ > 1. Prove that the
set of noninvertible operators on $V$ is not a subspace of $\mathcal{L}(V)$. (Hint: you have seen this
example in previous homeworks when dim $V$= 3.)

\uwave{Pf\enskip.}
\vspace{0.618 em}

Let $\mathcal{F} = \{f_i\}_{i=1}^n$ be a basis for $V$

$\forall T\enskip\in
\mathcal{L}(V): \exists!\enskip
[T]_{\mathcal{F}\leftarrow \mathcal{F}} \in $Mat$_{n \times n}(F)$ and
$\mathcal{L}(V) \cong$ Mat$_{n \times n}(F)$

Since, $T$ is invertible $\Rightleftarrow$ det$([T]) \neq 0$

So, $X = \{A \in $Mat$_{n×n}(\mathbb{F}) |\enskip $det$ A =0\}$ is
isomorphic to the set of noninvertible operators on $V$.

$E_{nn}$ is the $n \times n$ matrix with zeros everywhere except for a one at the entry on
the nth row and nth column.

det$(I_{n \times n} - E_{nn})$ $=$ tr$(I_{n \times n} -E_{nn})$ $= 1^{n-1}(0) = 0$

$\Rightarrow I -E_{nn} \in X$

det$(E_{nn})$ $=$ tr$(E_{nn})$ $= 0^{n-1}(1) = 0$

$\Rightarrow E_{nn} \in X$

$I_{n \times n} = I_{n \times n} + 0_{n \times n} = I_{n \times
  n}+(E_{nn} - E_{nn})  = (I_{n \times n} -E_{nn}) + E_{nn}$

det$(I_{n \times n}) = 1^n = 1$

So, $X$ is not closed under addition.

$X \not\trianglelefteq$ Mat$_{n \times n}(F)$

\vspace{0.618 em}
$\blacksquare$


\end{document}

%%% Local Variables:
%%% mode: latex
%%% TeX-master: t
%%% End:
